For the regular expression homework we have seen the youtube video that can be found here: \cite{leaverou1}. In this video \textbf{regular expressions} are taught from the absolute basics to some more advanced uses. \textbf{It is a very good video for beginners to watch and it} really demystifies most of the the weird constructions \textbf{\textit{used in regular expressions}}. Some good tips to take away from this presentations are:
\begin{itemize}
\item It is better to have false positives than false negatives
\item \textit{Do not make the regular expression too complicated, because the next time you will have to change it will be impossible}
\end{itemize}
Some example she gives are given in this code block \ref{lst:exampleregex}.

\begin{lstlisting}[language=bash,label=exampleregex,caption=Regex examples]
# Here are some examples she has shown in the video.
/ab/
# will match ab, abdawidojad, dbabsdab, helloabhello
/\bab\b/ 
# will only match ab from the previous given examples
/a(=?b)/
# will met all the a with b behind it.
/a{3}/
# will match exactly 3 a's so aaa
/a{3,}/
# will match 3 or more a's
\end{lstlisting}

This is a reference to my table. It is a very nice table and can be seen in table \ref{tab:testtable}

\begin{table}
\caption{ nice table.}
\label{tab:testtable}
\begin{tabular}{|l|l|l|}
\hline
hello & This & a \\ \hline
very & nice & table\\ \hline
\end{tabular}
\end{table}