

We are not going to talk about the subject, but we are going to talk about the article itself. We think it is a very good article, because it makes you think about possible solutions for problems you might encounter or already encountering.

The subject of the article can be interchanged with any other problem and you can use the same method the author used to try to solve it.
\textbf{All the problems he is addressing are mostly due to the reluctance of the people that are using makefiles.} The reluctance to try new things. The reluctance to investigate what is really going on and the assumption that the way things are is the best way.

In this article he gives multiple example of how people are trying to make patches for the problem they are experiencing, but \underline{never} directly addresses the real problem. For example people know that sometimes the the program does not compile the way they want to, because not all the dependencies are updated. So the solution that people has adopted and are still using is to do another make to update the dependencies and afterwards they think the compilation will be ok. But it is of course not guaranteed that the second time all the files will be up to date for all the other files. This is just one of the examples he gives, but it all amounts to the same problem. People are reluctant to try new things and to think about what the problem is and where does it come from. \cite{recursivemake}

We think this article really promotes scientific thinking and never take anything for granted. The author really has encountered some problems, noticed the similarity in the problems, then he took a step back to look at how actually the recursive method is working and why problems are often encountered. He finally proposes a solution and defend his opinion about the problems the users could think about his solution with proofs and references.

An example for this is that people used the argument that if you use just one makefile for everything it will never be able to fit the memory. This might have been the case a decade ago but now the memory is upgraded to the point that it does fit and the argument is not valid anymore. This is an example of people not thinking for themselves. If you just stop and think you would know this argument would not hold anymore.\textit{For my information please watch the video by D.Weeler}
\url{https://www.youtube.com/watch?v=e-uYBb554LU}

So the main things to learn from this article is to think about what you are doing. Take a step back and have a more scientific approach. Try new things out. Think about the way things are done now and wonder why it has been done this way in the past and then judge if the technology has evolved and if you can improve it. You can't fix what is fundamentally broken  .

